Nous avons entraîné les modèles unimodaux et le \textit{EARLY FUSION}, sur 2 epochs, car ils utilisent tous (sauf \textit{VIDEO}), 
un modèle pré-entraîné, comment BERT ou Wave2Vec. Nous avons ensuite gardé les poids de 
l'epoch qui avait la plus basse loss de validation.

Cependant, l'entraînement du modèle \textit{VIDEO}, dû a la grandeur des données et 
leurs temps de chargement dans la RAM, nous avons décidé de faire qu'une seule epoch
\footnote{Une epoch durait plus de 2h30 sur google colab avec un GPU Nvidia T4.}, et de ne pas l'intégrer dans les modèles multimodaux.

\subsection{Loss et poids}
Nous avons utilisé la fonction de coût \textit{crossentropy} pour l'entrainement de modèles. 
Cependant, du fait, du fait du grand déséquilibré des classes (voir Table~\ref{tab: label distribution}), nous avons ajouté des poids 
sur les classes dans la fonction crossentropy. En effet, sans ces poids, les modèles avaient tendance à prédire toujours des $0$, ce 
qui leurs permetait d'optenir $80\%$ d'accuracy sans dificultés. Nous avons alors cherché des poids à mettre dans la loss de telle sorte 
que les modèles ne prédisaient pas systématiquement qu'un seul label
\footnote{Si on met un poids trop grand pour le label 1, même s'il y a que $20\%$ de $1$, les modèles vont prédire que des $1$.}.
Nous avons alors trouvé et appliqué les poids $w_0 = 1$ et $w_1=3.9$. La loss se calcule alors avec l'équation~\ref{eq: loss}
(avec des notations standards)~:

\begin{equation}
    L(x, y) = \frac{-1}{N} \sum_{n=0}^{N} w_0 (1-y_n) \log(1-x_n) + w_1 y_n \log(x_n)
    \label{eq: loss}
\end{equation}

\subsection{Métriques}
Pour comparer les résultats obtenus, nous avons utilisé la loss, mais aussi d'autres métriques,
comme l'accuracy, la précision, le rappel et le $f_1$-score.

Du fait des déséquilibres de classes, l'accuracy n'est pas les métriques la plus cruciale. On a donc opté pour 
calculer aussi la précision, le rappel et le $f_1$-score. Pour ces métriques, on a choisi de les calculer 
sur chacune des classes et d'en faire la moyenne. Ainsi un rappel de $0.3$ signifiera que la moyenne du rappel sur
la classe $0$ et la classe $1$ fasse $0.3$. Nous avons choisi de faire cela, car si l'on regardait seulement 
le rappel sur la classe $1$, le TRUE POSITIF (TP) est très souvent égale à $0$ et donc le rappel est nul. De même 
pour la précision et de $f_1$-score, ce qui nous apportait moins d'information.